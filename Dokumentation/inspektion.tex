\section{Aufgabe}
		\subsubsection{Was messen wir?}
		Wir haben drei Messstationen(Vielleicht mehr in der Zukunft), die immer kontrolliert werden müssen.
		Wir werden eine Kontrolle und Analyse von Software und Hardware machen. 
		\subsubsection{Wann messen wir?}
		Wir messen die Temperatur, deswegen müssen wir regelmassig  um 7 Uhr messen. Es wird noch viele unterschiedliche Hardware/Software Testen geben. Die Testen werden täglich durchgeführt.
		\subsubsection{Wie oft messen wir?}
		Abhängig von der Art von Test werden 3 Testen durchgeführt. Nachdem wir um 7 Uhr gemessen haben, wird routine Kontrolle geben. Die Routinekontrolleliste befindet sich im Section 2.
		\newpage 
\section{Routine}
		\subsubsection{Was ist eine Routine?}
		Eine Routine ist Gesamtliste von Testen, die täglich durchgeführt werden. Bei der Routine werden die Zeit und Typ von Kontrolle definiert. Bei der Tabelle wird eine Check-Spalte, wo man, ob alles in Ordnung ist, sieht, geben. 
		\subsubsection{Routine Tabelle}
		\begin{longtable}{|l|l|l|p{7cm}|}
			\hline Uhr & Typ & Check & Problem
			\endhead
			\hline 07:00 & Full-Check & - & -\\
			\hline  & 1.Rasberry Pi &  & \\
			\hline  & 2. WLAN Verbindungen &  & \\
			\hline  & 3. Schnittstellen &  & \\
			\hline  & 4. Temperatur Messungen &  & \\
			\hline  & 5. Software/Programm &  & \\
			\hline  & 6. Website &  & \\
			\hline  & 7. Datenbank &  & \\
			\hline 13:00 & Software-Check & - & -\\
			\hline  & 1. Temperatur Messungen &  & \\
			\hline  & 2. Software/Programm  &  & \\
			\hline  & 3. Datenübertragung Korrektheit &  & \\
			\hline 			\\

			\caption{Routine Tabelle}
			\label{lt:routine}
		\end{longtable}