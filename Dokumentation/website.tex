\pagestyle{fancy}
\fancyfoot[R]{A.Terzeta, A.Isufi, A.Sheldija}
In diesem Kapitel geht es um die Verbindung des PC-s mit dem Server und der Website.
Die Werte, die von den Messkisten geschickt werden, werden in einem Datenbank gespeichert. Jede Kiste hat einen eigenen ID, Raum, wo sie ist und die MAC-Adresse der Raspberry PI. Die server-seitige Programmierung wird mit PHP gemacht und der client-seitige Programmierung (die Website) mit HTML,CSS f\"ur Design der Website und PHP, die uns hilft, die Website mit dem Webserver kommunizieren zu k\"onnen. 
	\section{Website}
	Die Webseite ist ziemlich einfach. Es besteht auf zwei Pages: Home und Messkisten. 
	Bei Home(Abb.\ref{fi:Web1}) werden die Temperaturdaten eingezeigt. Es gibt ein kleines Formular, das den Graph ver\"andert, abh\"anglich von welcher Klasse, Einheit oder Datum der Benutzer eingibt.	
	Bei Messkiste (Abb.\ref{fi:Web2}) werden die Daten der Messkisten angezeigt und auch ver\"andert werden. Wenn eine neue Messkiste angelegt wird, hat sie keine Raum-bezeichnung. Das wird \"uber die Website gemacht, mit dem 'Edit' Button am Ende der jeden Zeile. 
		\begin{figure}[ht]
		\centering
		\includegraphics[scale=0.6]{./bilder/page1Verbessert.png}
		\caption{Website-Home}
		\label{fi:Web1}
		\end{figure} 
	\begin{figure}[ht]
		\centering
		\includegraphics[scale=0.4]{./bilder/Page_2.png}
		\caption{Website-Messkiste}
		\label{fi:Web2}
	\end{figure} 
	\section{Datenbankdesign}
	Eine Datenbank wird erstellt, die auf einem Webserver l\"auft. Der Server wird mithilfe einer Raspberry PI simuliert. Der Server hat eine statische IP-Adresse und muss 24 Stunden online sein. Die Verbindung mit der Datenbank wird \"uber WLAN durchgef\"uhrt. Sehen Sie auf die Abb. \ref{fi:ERD}
	\subsection{Struktur der Datenbank}
	Die Datenbank hat zwei Tabellen. Die Tabellen werden mit einem Fremdschl\"ussel miteinander verbunden. Die erste Tabelle hei{\ss}t Messkiste. Diese Tabelle hat 4 Spalten. 
	\begin{itemize}
		\item Die erste Spalte ist 'ID', mit der die Messkiste identifiziert wird. ID ist unique und jede Messkiste hat eine eigene ID. Die ID wird selbst f\"ur jede neue Messkiste um eins erh\"oht. Datentyp von ID ist 'int' und darf keine Null-Werte haben. 
		\item Die zweite Spalte ist 'Raum', die einen Datentyp varchar(50) hat. Das bedeutet, dass Name des Raumes muss maximum 50 Buchstaben enthalten. Jede Messkiste hat einen Taster. Wenn sie montieren werden, wird dieser Taster gedr\"uckt und wird in der Website gezeigt, dass der Taster dieser Messkiste mit diesem ID ist gedr\"uckt und wird gespeichert, in welchem Raum die Messkiste ist.
		\item Die dritte Spalte ist 'Status'. In dieser Spalte wird gespeichert, ob die Messkiste mit der Datenbank verbunden ist. 
		\item Die vierte Spalte ist 'MAC-Adresse'. Der Datentyp von dieser Spalte ist varchar(12). MAC-Adresse ist von Raspberry PI. In der Datenbank werden die Bindestriche von MAC-Adresse nicht gespeichert z.B. 'AABBCCDDEEFF'.
		\begin{figure}
			\centering
			\includegraphics[scale=0.7]{./bilder/db.png}
			\caption{ERD-Diagramm}
			\label{fi:ERD}
		\end{figure} 
		\end{itemize}
	Die zweite Tabelle hei{\ss}t 'Messwerte' und hat 5 Spalten. Diese Tabelle wird mit der ersten Tabelle durch einen Fremdschl\"ussel verbundet.
	\begin{itemize}
		\item Die erste Spalte ist 'ID', mit der die Messwert identifiziert wird. ID ist unique und jede Messkiste hat eine eigene ID. Die ID wird selbst f\"ur jede neue Messkite um eins erh\"oht. Datentyp von ID ist 'int' und darf keine Null-Werte haben. 
		\item Die zweite Spalte ist 'ID\_Messkiste'. Diese Spalte dient als Fremdschl\"ussel, der die ID von der Spalte 'Messkiste' referenziert. Datentyp ist 'int'.
		\item Die dritte Spalte ist 'Datum'. In dieser Spalte wird gespeichert, wann die Werte vom Sensor im Webserver gekommen sind. Datentyp ist 'date'.
		\item Die vierte Spalte ist 'Temp'. Der Datentyp ist 'double'. Die Temperatur wird vom Sensor gemessen und in der Datenbank gespeichert.
		\item Die f\"unfte Spalte ist 'Time'. Diese Spalte speichert die genaue Uhrzeit. Diese Uhrzeit werden wir brauchen, wenn wir die Temperaturen von einem Tag in Website angezeigt wird.  
	\end{itemize}
	\subsection{Installation}
	Zuerst m\"ussen die packages installiert werden, damit mysql-server und php bzw.Apache Server laufen kann und der Benutzer sie verwenden kann. "apache2, mysql-server,php libapache2-mod-php php-mysql". Alles, was mit php bzw. mit Websiten yu tun haben, sind im Verzeichnis /var/www. Wir erstellen einen neuen Verzeichnis(website) da und in diesem Verzeichnis speichern wir die PHP-Datei und alle anderen, die mit der Website zu tun haben z.B. Bilder, CSS-Datei. WWW-Data ist der Standard-Benutzer f\"ur WWW und wir brauchen den neuen Verzeichnis(website) diese Rechte geben. Befehl ist: chown www-data <Das Verzeichnis> und wir brauchen auch den Befehl 'chmod 2750 <Das Verzeichnis>'. Dann \"offnen wir die Datei 000-default.conf in dem Verzeichnis /etc/apache2/sites-available und erstellen da ein Alias + ein Directory. Nachdem alle Schritte fertig sind, \"ofnnen wir die PHP-Datei und arbeiten da. 
	Der Grund-Design der html-seite mit metacharset="UTF-8". Da verbinden wir auch die CSS-Datei. Im Body-Bereich machen wir ein DIV, wo der Graph gespeichert wird. Wir haben daf\"ur 'Charts' verwendet(CanvasJs.Chart). Im PHP-Bereich verbinden wir uns mit dem Webserver mithilfe des PDOs. Nachdem wir uns mit dem verbinden, erstellen wir einige 'select-statement', welche wir brauchen, um den Graph mit Temperatur ausf\"ullen. 
